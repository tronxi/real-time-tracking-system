\chapter{Conclusions and Future Work}
\label{cap:conclusions}

The main objective set at the beginning of this work, to create a real-time data 
visualization platform that is reusable and decoupled from specific hardware, has been 
achieved. This was accomplished through the analysis of existing solutions and the design 
of an architecture focused, from its earliest stages, on hardware independence and 
straightforward integration into other projects.

The modular architecture of the system facilitates its maintenance and evolution. Each 
component (data acquisition, transmission, backend, frontend) can be 
modified independently without affecting the rest of the system, which allows for partial 
reuse.

The construction of a CanSat has served as a case study to validate the platform, 
confirming that there are no dependencies between the visualization system and the 
hardware. This example demonstrates that the solution can be easily adapted to different 
sensor configurations, protocols, or CanSat architectures.

The development of the complete workflow, from data acquisition to real-time 
visualization, made it possible to identify critical points and optimize transmission 
times, ensuring event synchronization in real environments. As a result, the project has 
produced a flexible platform, useful not only as a standalone solution but also as a 
foundation for future developments in projects with similar data visualization needs.

During the development, several challenges also had to be addressed. One of the most 
significant was the Raspberry Pi Zero~2~W, which showed limitations in reliably managing 
real-time video streaming. To make it work, it was necessary to reduce the resolution to 
640×480 pixels and the frame rate to 15~fps, as well as to use ffmpeg with 
hardware-accelerated H.264 encoding. These optimizations allowed the video stream to be 
transmitted continuously to the RTMP server, ensuring uninterrupted visualization on the 
web application.

Another issue was related to the connection of the BN-880 GNSS receiver. This Raspberry 
model provides only one UART port accessible through the GPIOs, which in this case was 
already used by the LoRa module. To add the GNSS, it was necessary to use a USB–UART 
adapter based on the CP2102 chip. Due to the limited available space, the adapter had to 
be physically modified by removing the original connector and soldering a more compact 
micro-USB. This adaptation made it possible to integrate the GNSS into the system without 
exceeding the maximum dimensions of the CanSat.

Finally, several lines of improvement have been identified for future work:

\begin{itemize}
    \item Expand the number of charts in the web application, adding visualizations for specific parameters.
    \item Add a telecommand system to manage the CanSat remotely. This would allow enabling or disabling sensors or changing parameters such as the data transmission interval at runtime.
    \item Introduce an authentication system in the web application to protect sensitive data.
    \item Support simultaneous visualization of multiple CanSats. At present, the origin of events is not distinguished, which prevents monitoring several devices at the same time.
    \item Extend the development of the CanSat beyond electronics and software with a 3D-printed enclosure and a parachute to enable real launch operations.
\end{itemize}



