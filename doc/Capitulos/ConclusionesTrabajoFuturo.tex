\chapter{Conclusiones y Trabajo Futuro}
\label{cap:conclusiones}

El objetivo principal planteado al inicio del trabajo —crear una plataforma de 
visualización de datos en tiempo real, reutilizable y desacoplada de hardware 
específico— ha sido alcanzado. Esto se consiguió gracias al análisis de otras 
soluciones existentes y al diseño de una arquitectura centrada, desde sus primeras 
fases, en la independencia del hardware y en su integración sencilla en otros proyectos.

La arquitectura modular del sistema facilita su mantenimiento y evolución. Cada 
componente (captura de datos, transmisión, \emph{backend}, \emph{frontend}) puede 
modificarse de manera independiente sin afectar al resto, lo que permite su 
reutilización parcial.

La construcción de un CanSat ha servido como caso de ejemplo para validar la 
plataforma, confirmando que no existen dependencias entre el sistema de visualización 
y el hardware. Este ejemplo demuestra que la solución puede adaptarse fácilmente a 
configuraciones de sensores, protocolos o arquitecturas de CanSat distintas.

El desarrollo del flujo completo, desde la adquisición de datos hasta su visualización 
en tiempo real, permitió identificar puntos críticos y optimizar los tiempos de 
transmisión, garantizando la sincronización de eventos en entornos reales. Como 
resultado, el trabajo ha generado una plataforma flexible, útil no solo como solución 
cerrada, sino también como base para futuros desarrollos en proyectos con necesidades 
similares de visualización de datos.

Durante el desarrollo surgieron también algunas dificultades que fue necesario 
superar. Una de las más relevantes estuvo en la Raspberry Pi Zero~2~W, que presentaba 
limitaciones para gestionar de forma estable el envío de vídeo en tiempo real. Para 
hacerlo funcionar fue necesario reducir la resolución a 640×480 píxeles y la tasa de 
fotogramas a 15~fps, además de utilizar ffmpeg con codificación por hardware en 
H.264. Estas optimizaciones permitieron que el flujo de vídeo se transmitiera de manera 
continua hacia el servidor RTMP, garantizando una visualización sin interrupciones.

Otro problema estuvo relacionado con la conexión del receptor GNSS BN-880. Este modelo de Raspberry solo dispone de un puerto UART accesible desde los GPIO, utilizado en este caso  por el módulo LoRa. Para poder añadir el GNSS fue necesario usar un adaptador 
USB–UART basado en el chip CP2102. Debido al espacio reducido disponible, hubo que 
modificar físicamente el adaptador, retirando el conector original y soldando un 
micro-USB más compacto. Esta adaptación permitió integrar el GNSS en el sistema sin sobrepasar las dimensiones máximas del CanSat.

Finalmente, se han identificado varias líneas de mejora en las que trabajar en el 
futuro:

\begin{itemize}
    \item Aumentar el número de gráficas disponibles en la aplicación web, incorporando visualizaciones específicas para determinados parámetros.
    \item Implementar un sistema de envío de telecomandos al CanSat para manejar su configuración de manera remota, esto permitiría activar o desactivar sensores o modificar parámetros como el intervalo de envío de datos directamente en tiempo de ejecución.
    \item Implementar un sistema de autenticación en la aplicación web para proteger posibles datos sensibles.
    \item Añadir soporte para visualizar múltiples CanSat al mismo tiempo. Actualmente no se distingue el origen de los eventos, por lo que no es posible monitorizar varios dispositivos simultáneamente.
    \item El desarrollo del CanSat se ha centrado en la parte electrónica y de software, se podría crear una carcasa 3D y un paracaídas para poder usarlo en lanzamientos reales.
\end{itemize}








