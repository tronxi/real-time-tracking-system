\chapter{Fundamentos teóricos}
\label{cap:fundamentos_teoricos}
Este capítulo se enfoca en analizar en profundidad los aspectos técnicos necesarios para llevar a cabo el desarrollo del CanSat.
Se incluye una comparativa entre los distintos microcontroladores y microcomputadores disponibles en el mercado, evaluando su precio, disponibilidad y adecuación a los requisitos técnicos del proyecto.
Además, se analizan los sensores y módulos de comunicación más comunes, así como los principales protocolos utilizados para la comunicación entre el microcontrolador y los sensores, como I2C y UART.
También se estudian distintas opciones para la retransmisión de vídeo en tiempo real desde el CanSat.
Por último, se exploran soluciones para la visualización de los datos a través de una interfaz web en tiempo real.


\section{Comparativa de microcontroladores y microcomputadores}
Uno de los componentes principales de un CanSat o de cualquier sistema embebido en general es su microcontrolador o microcomputador,
es el encargado de comunicarse con los sensores, procesar los datos y transmitirlo a través del módulo de comunicación,
también es el encargado de procesar el video y retransmitirlo en tiempo real(si el hardware lo permite).

Primero conviene aclarar la diferencia entre microcontrolador y microcomputador:
\begin{itemize}
    \item \textbf{Microcontrolador:} Circuito integrado que combina procesador, memoria y periféricos de entrada/salida en un solo chip.
    Está diseñado para realizar tareas específicas con bajo consumo energético y recursos limitados.
    Es común en aplicaciones como lectura de sensores, control de motores o gestión de comunicaciones básicas.
    Ejemplos comunes son Arduino Uno o ESP32
    \item \textbf{Microcomputador:} Sistema completo en una sola placa (Single-Board Computer) que integra procesador, memoria, almacenamiento y puertos de expansión.
    Es capaz de ejecutar sistemas operativos completos (como Linux) y realizar tareas más complejas, como procesamiento de imágenes, servidor web o interfaces gráficas.
    Un ejemplo típico es la Raspberry Pi Zero.
\end{itemize}

Actualmente, en el mercado se pueden encontrar varios de estos microcontroladores o microcomputadores de bajo coste, tamaño y peso reducido y bajo consumo.
En esta sección nos vamos a centrar en el análisis de tres de las opciones más populares:
\begin{itemize}

    \item \textbf{\cite{raspberrypi_zero2}:}
    La Raspberry Pi Zero 2 es un Single Board Computer de bajo costo lanzado en Reino Unido por la~\cite{raspberrypi_foundation},
    de toda la familia Raspberry Pi se ha elegido el modelo Zero 2 por ser el de tamaño más reducido pero con una potencia adecuada para el desarrollo de un proyecto como el CanSat.
    Las características más relevantes para este proyecto son:
    \begin{itemize}
        \item CPU Arm Cortex-A53 de cuatro núcleos y 64 bits a 1 GHz.
        \item SDRAM de 512 MB.
        \item LAN inalámbrica de 2,4 GHz 802.11 b/g/n.
        \item Bluetooth 4.2, Bluetooth Low Energy (BLE), antena integrada.
        \item Ranura para tarjeta microSD.
        \item Conector de cámara CSI-2.
        \item Cabecera GPIO de 40 pines para conexión de periféricos.
        \item 1 × SPI
        \item 2 × interfaces I²C
        \item 1 × UART
        \item Consumo típico de energía: entre 0.7\,W y 1.5\,W dependiendo de la carga de trabajo.
        \item Precio aproximado: 15–20€.
    \end{itemize}
    \begin{figure}[h]
        \centering
        \includegraphics[width=0.8\textwidth]{Imagenes/Bitmap/pizero2gpio}
        \caption{Distribución de pines GPIO en Raspberry Pi Zero 2}
        \label{fig:pizer2_gpio}
    \end{figure}
    Como se puede ver en las características la Raspberry Pi Zero 2 cumple con los requisitos necesarios para este proyecto,
    cuenta con un procesador y memoria adecuados, soporte para cámara (útil para la retransmisión de video en tiempo real), conectividad inalámbrica mediante WiFi
    y compatibilidad con los protocolos I\textsuperscript{2}C y UART mediante los pines GPIO necesarios para la conexión directa de sensores.

    \item \textbf{\cite{esp32}:}
    Es un microcontrolador de bajo coste desarrollado por Espressif Systems que combina bajo coste con buena capacidad de procesamiento y conectividad inalámbrica integrada,
    lo que lo convierte en una de las opciones más populares para proyectos embebidos, incluyendo CanSat.

    A diferencia de un microcomputador como la Raspberry Pi, el ESP32 no ejecuta un sistema operativo generalista,
    pero su bajo consumo energético y la integración de múltiples periféricos lo hacen convierten en muy buena opción cuando se busca eficiencia y simplicidad del sistema.

    Las características más relevantes para este proyecto son:
    \begin{itemize}
        \item CPU dual-core Tensilica Xtensa LX6 a 240\,MHz.
        \item 520\,KB de SRAM interna.
        \item Memoria flash externa: normalmente 4\,MB (dependiendo del modelo).
        \item Conectividad Wi-Fi 802.11 b/g/n.
        \item Bluetooth 4.2 y BLE.
        \item 4 × SPI
        \item 2 × interfaces I²C
        \item 3 × UART
        \item Hasta 34 pines GPIO (según versión del módulo).
        \item Consumo típico: entre 0.2\,W y 0.6\,W, dependiendo del modo de operación.
        \item Precio aproximado: 4–8€~\cite{esp32}.
    \end{itemize}
    \begin{figure}[h]
        \centering
        \includegraphics[width=0.8\textwidth]{Imagenes/Bitmap/esp32gpio}
        \caption{Distribución de pines GPIO en ESP32}
        \label{fig:esp32gpio}
    \end{figure}

    Gracias a su bajo consumo, potencia y multiples conexiones, el ESP32 permite integrar sensores fácilmente a través de interfaces estándar y puede encargarse tanto de la adquisición como de la transmisión de datos por radio o Wi-Fi.
    Además, su bajo consumo lo hace especialmente adecuado para sistemas alimentados por batería en entornos con restricciones energéticas.
    También existen variantes como el ESP32-CAM que integran una cámara de tipo OV2640, lo que permite capturar imágenes y transmitir video mediante Wi-Fi
    aunque con un rendimiento y resolución inferior a la de Raspberry Pi.

    \item \textbf{\cite{arduino_nano}:}
    Es un microcontrolador compacto de bajo coste basado en el chip ATmega328P, es el más utilizado en entornos educativos gracias a su simplicidad,
    por lo que cuenta con una amplia comunidad detrás y desarrollo de librerias.
    A diferencia de la Raspberry Pi Zero 2 o el ESP32, el Arduino Nano no cuenta con conectividad inalámbrica ni capacidad de procesamiento avanzada,
    pero es suficiente para gestionar sensores básicos y transmitir datos mediante un módulo externo de radio.

    Las características más relevantes para este proyecto son:
    \begin{itemize}
        \item Microcontrolador ATmega328P.
        \item Frecuencia de reloj: 16\,MHz.
        \item Memoria flash: 32\,KB (2\,KB utilizados por el bootloader).
        \item SRAM: 2\,KB.
        \item EEPROM: 1\,KB.
        \item 22 pines GPIO (14 digitales, 8 analógicos).
        \item 1 × interfaz I²C.
        \item 1 × UART.
        \item 1 × SPI.
        \item Consumo típico: entre 0.05\,W y 0.2\,W.
        \item Precio aproximado: 27€ en la web oficial.
    \end{itemize}

    \begin{figure}[h]
        \centering
        \includegraphics[width=0.7\textwidth]{Imagenes/Bitmap/arduinoNanogpio}
        \caption{Distribución de pines en Arduino Nano}
        \label{fig:arduino_nano_gpio}
    \end{figure}

    Aunque no tiene las capacidades de procesamiento de una Raspberry Pi ni la conectividad integrada del ESP32, el Arduino Nano puede ser una solución válida para CanSat muy simples,
    en los que se priorice el consumo mínimo y no se necesiten funcionalidades avanzadas como WiFi o procesamiento de vídeo, sin embargo, al no tener soporte para cámaras,
    no cumple con los requisitos técnicos necesarios para este proyecto.

\end{itemize}

\begin{table}[h]
    \centering
    \footnotesize
    \resizebox{1.1\textwidth}{!}{
        \begin{tabular}{|l|l|l|l|}
            \hline
            \textbf{Característica} & \textbf{Raspberry Pi Zero 2} & \textbf{ESP32}           & \textbf{Arduino Nano}   \\
            \hline
            Procesador              & ARM Cortex-A53 (4×, 1 GHz)   & Xtensa LX6 (2×, 240 MHz) & ATmega328P (1×, 16 MHz) \\
            \hline
            Memoria                 & 512 MB SDRAM                 & 520 KB SRAM + 4 MB Flash & 2 KB SRAM + 32 KB Flash \\
            \hline
            Wi-Fi                   & Sí                           & Sí                       & No                      \\
            \hline
            Bluetooth               & 4.2 + BLE                    & 4.2 + BLE                & No                      \\
            \hline
            SPI                     & 1                            & 4                        & 1                       \\
            \hline
            I²C                     & 2                            & 2                        & 1                       \\
            \hline
            UART                    & 1                            & 3                        & 1                       \\
            \hline
            Compatibilidad cámara   & CSI (cámara oficial)         & OV2640 (ESP32-CAM)       & No                      \\
            \hline
            Consumo típico          & 0.7--1.5 W                   & 0.2--0.6 W               & 0.05--0.2 W             \\
            \hline
            Precio estimado         & 15--20 €                     & 4--8 €                   & 25--30 €                \\
            \hline
        \end{tabular}}
    \caption{Comparativa de microcontroladores y microcomputadores}
    \label{tab:comparativa-mcus}
\end{table}


\section{Interfaces de comunicación serie}
Una vez analizados sobre los distintos microcontroladores y microprocesadores que existen en el mercado para este tipo de proyectos,
es importante entender los distintos tipos de interfaces de comunicación que utilizan para interactuar con sensores y otros módulos externos y como funcionan.
En esta sección se presentan tres de los más relevantes: SPI, I²C y UART

\begin{itemize}
    \item \textbf{SPI:} La interfaz SPI (Serial Peripheral Interface)~\cite{dhaker_spi} es una interfaz síncrona y full dúplex basada en una arquitectura maestro-esclavo,
    los dos dispositivos, maestro y esclavo pueden transmitir datos simultáneamente sincronizados con una señal de reloj.
    La interfaz SPI utiliza cuatro señales:
    \begin{itemize}
        \item Señal de reloj (CLK)
        \item Selección de chip (CS)
        \item Salida del maestro hacia el esclavo (MOSI)
        \item Salida del esclavo hacia el maestro (MISO)
    \end{itemize}
    \begin{figure}[h]
        \centering
        \includegraphics[width=0.8\textwidth]{Imagenes/Bitmap/spi}
        \caption{Configuración SPI con un maestro y un esclavo}
        \label{fig:spi}
    \end{figure}
    El maestro genera la señal de reloj y controla el intercambio de datos.
    El pin CS selecciona el estado activo y los pines MISO y MOSI transportan datos en ambas direcciones.
    Para iniciar la comunicación, el maestro activa el pin CS y empieza a emitir la señal de reloj, al ser una interfaz full dúplex,
    maestro y esclavo pueden enviar y recibir datos simultáneamente.
    Esta interfaz tiene un solo maestro y puede tener uno o múltiples esclavos.
    En configuraciones con múltiples esclavos se pueden conectar de dos modos:
    \begin{itemize}
        \item \textbf{Modo regular:} Cada nodo tiene su propia línea de CS
        \item \textbf{Modo cadena (daisy-chain):} Todos los nodos comparten el mismo reloj y CS y los datos se propagan de un esclavo al siguiente.
        De esta manera se reduce el número de GPIO necesarios en el maestro, aunque aumenta el número de ciclos de reloj requeridos para llegar a cada esclavo.
    \end{itemize}
    La velocidad de transferencia puede variar dependiendo del hardware utilizado, lo habitual es entre 1 y 10Mbps, aunque algunos dispositivos permiten velocidades superiores.

    \item \textbf{I²C:} La interfaz I²C (Inter-Integrated Circuit)~\cite{i2c_specification} es un bus de comunicación síncrono y half dúplex basado en una arquitectura maestro-esclavo, que utiliza solo dos líneas para comunicarse con múltiples dispositivos,
    originalmente fue desarrollada por Philips Semiconductors en 1982.
    Esta interfaz utiliza solo dos señales:
    \begin{itemize}
        \item Línea de datos (SDA)
        \item Línea de reloj (SCL)
    \end{itemize}
    \begin{figure}[h]
        \centering
        \includegraphics[width=0.8\textwidth]{Imagenes/Bitmap/i2c}
        \caption{Configuración I²C con un maestro y varios esclavos}
        \label{fig:i2c}
    \end{figure}
    Ambas líneas son bidireccionales aunque al ser half dúplex la comunicación se realiza en un sentido a la vez.
    Un dispositivo actúa como maestro, iniciando la comunicación y generando la señal de reloj, mientras que uno o más esclavos responden.
    Cada dispositivo conectado al bus tiene una dirección única.

    La comunicación se inicia cuando el maestro envía una condición de inicio, la dirección de uno de los dispositivos conectados al bus y un bit indicando si va a leer o escribir.
    Después de transmitir cada byte, el receptor envía una señal de reconocimiento (ACK). La transmisión termina cuando se envía una condición de parada.

    El bus I²C permite velocidades de transferencia de hasta 100 kbit/s en modo estándar (Standard-mode), 400 kbit/s en modo rápido (Fast-mode), 1 Mbit/s en modo fast-mode plus (Fm+), y hasta 3.4 Mbit/s en modo high-speed (Hs-mode), dependiendo de las capacidades del hardware.

    \item \textbf{UART:} La interfaz UART (Universal Asynchronous Receiver-Transmitter)~\cite{infineon_uart} es una interfaz de comunicación asíncrona basada en una arquitectura punto a punto,
    que permite la transmisión de datos en serie entre dos dispositivos.
    A diferencia de las interfaces anteriores, que eran síncronas, UART es asíncrona, por lo que no utiliza una señal de reloj compartida,
    sino que cada dispositivo funciona con una velocidad de transmisión acordada previamente y común para los dos (baud rate).
    UART emplea dos líneas de comunicación:
    \begin{itemize}
        \item Transmisión de datos (TX)
        \item Recepción de datos (RX)
    \end{itemize}
    \begin{figure}[h]
        \centering
        \includegraphics[width=0.8\textwidth]{Imagenes/Bitmap/uart}
        \caption{Ejemplo de conexión UART}
        \label{fig:uart}
    \end{figure}
    La línea TX de un dispositivo debe ir conectada a la línea RX del otro dispositivo, y la línea RX a TX.

    La comunicación es full dúplex, permitiendo la transmisión y recepción de datos de forma simultánea.
    Cada byte se transmite en una trama que incluye un bit de inicio (start bit), los bits de datos (normalmente 8), un bit opcional de paridad, y uno o más bits de parada (stop bits).

    Las velocidades de transmisión habituales van desde 9600 hasta 115200 baudios, aunque se pueden alcanzar velocidades de hasta 1 o 2 Mbps, dependiendo del hardware.

\end{itemize}

\begin{table}[h]
    \centering
    \footnotesize
    \renewcommand{\arraystretch}{1.1}
    \begin{tabular}{|l|c|c|c|}
        \hline
        \textbf{Característica} & \textbf{SPI}               & \textbf{I²C}               & \textbf{UART}       \\
        \hline
        Tipo de comunicación    & Síncrona                   & Síncrona                   & Asíncrona           \\
        \hline
        Arquitectura            & Maestro-esclavo            & Maestro-esclavo            & Punto a punto       \\
        \hline
        Número de líneas        & 4 (CLK, CS, MOSI, MISO)    & 2 (SDA, SCL)               & 2 (TX, RX)          \\
        \hline
        Full/Half dúplex        & Full dúplex                & Half dúplex                & Full dúplex         \\
        \hline
        Número de dispositivos  & 1 maestro, varios esclavos & 1 maestro, varios esclavos & Solo dos            \\
        \hline
        Velocidad típica        & 1–10 Mbps                  & 100 kbit/s – 3.4 Mbit/s    & 9.6 kbit/s – 3 Mbps \\
        \hline
        Control de dirección    & Señal CS por esclavo       & Dirección en el protocolo  & No necesario        \\
        \hline
    \end{tabular}
    \caption{Comparativa entre las interfaces SPI, I²C y UART}
    \label{tab:comparativa_interfaces}
\end{table}


\section{Tecnologías de comunicación por radiofrecuencia en sistemas embebidos}
Dentro de los sistemas embebidos y en particular en sistemas tipo CanSat es muy importante la comunicación inalámbrica entre el satélite y la estación de tierra.
De entre las diversas opciones de comunicación inalámbrica, en este apartado vamos a estudiar las distintas opciones que hay en el mercado de comunicación por radiofrecuencia.
Este tipo de comunicación nos ofrece un largo alcance manteniendo el bajo consumo, un aspecto muy relevante en un CanSat.

Las tres opciones de comunicación por radiofrecuencia que vamos a estudiar son:

\begin{itemize}
    \item \textbf{LoRa:} La tecnología (Long Range)~\cite{augustin_lora} permite una comunicación inalámbrica de largo alcance con un bajo consumo energético,
    forma parte de las LPWAN (Low-Power Wide-Area Network) redes orientadas a aplicaciones de internet de las Cosas (IoT).

    Técnicamente, Lora funciona utilizando una tecnología de modulación de espectro ensanchado basada en chirp spread spectrum (CSS).
    Cada símbolo(conjunto de bits transmitidos como una única unidad) se transmite mediante un único chirp, es decir,
    una señal cuya frecuencia varía linealmente (aumentando o disminuyendo) a lo largo del tiempo.
    El número de bits por símbolo se determina por el Spreading Factor (SF), de modo que un SF de 7 bits codifica 7 bits por símbolo,
    un SF de 10 codifica 10 bits, y así sucesivamente.

    Además, LoRa incorpora un esquema de corrección de errores hacia adelante (Forward Error Correction, FEC), que mejora la fiabilidad de la comunicación en entornos con ruido o interferencias.
    Este sistema añade bits redundantes a los datos transmitidos, permitiendo al receptor detectar y corregir errores sin necesidad de retransmisión.
    La cantidad de redundancia depende de la tasa de codificación (Coding Rate, CR), que puede configurarse entre 4/5 y 4/8.
    Esta tasa indica cuántos bits útiles se transmiten por cada bloque de bits totales.
    Por ejemplo, una tasa de 4/5 significa que por cada 5 bits enviados, 4 contienen información útil y 1 es redundante para corrección de errores.
    De forma similar, una tasa de 4/8 implica que solo 4 de cada 8 bits son datos útiles y los 4 restantes son bits de corrección.
    Cuanto menor es la tasa, mayor es la redundancia y, por tanto, mayor la robustez frente a errores, aunque la velocidad efectiva de transmisión es menor.

    Opera en bandas de frecuencia no licenciadas, como 433 MHz, 868 MHz (Europa) o 915 MHz (América),
    esto permite su uso sin coste dentro de este espectro.

    El alcance varía dependiendo del entorno, desde cientos de metros en entornos urbanos con muchas interferencias hasta más de 15 kilómetros en entornos abiertos y despejados.

    \item  \textbf{XBee:} Los módulos XBee ~\cite{digi_xbee_15_4} han sido desarrollados por Digi International y permiten establecer comunicaciones inalámbricas de corto a medio alcance
    con bajo consumo energético.
    Utilizan el estándar IEEE 802.15.4 para las capas física y MAC, lo que les proporciona interoperabilidad con otros dispositivos compatibles con este estándar.

    Estos módulos operan en la banda ISM de 2.4GHz utilizando modulación DSSS (Direct Sequence Spread Spectrum) con O-QPSK (Offset Quadrature Phase Shift Keying), alcanzando una tasa de transferencia aérea de 250kbps.

    La técnica \textit{DSSS} consiste en dispersar cada bit de datos sobre una secuencia de mayor ancho de banda mediante una secuencia pseudoaleatoria (chipping code), lo que proporciona robustez frente a interferencias y permite que múltiples transmisiones compartan el mismo canal sin colisiones significativas.

    Por su parte, \textit{O-QPSK} es una variante de la modulación por desplazamiento de fase en cuadratura (QPSK), que introduce un desfase temporal entre las componentes en fase y en cuadratura para reducir los cambios bruscos en la señal transmitida,
    mejorando así la eficiencia espectral y reduciendo la probabilidad de errores durante la demodulación.

    Digi ha desarrollado interfaces de configuración y modos de operación propios, como el modo API, que encapsula datos y comandos en tramas estructuradas para control avanzado,
    además del modo transparente, que permite una comunicación directa punto a punto.

    XBee permite topologías de red como punto a punto o estrella.
    Cada módulo tiene una dirección MAC única de 64 bits y puede asignarse una dirección corta de 16 bits para identificación dentro de la red.
    La comunicación con el microcontrolador suele realizarse mediante UART, con velocidades configurables de hasta 1Mbps.

    En condiciones óptimas, el alcance puede variar entre 30 y 100 metros en interiores, y superar los 300 metros en exteriores.
    \item \textbf{APC220:} El módulo APC220~\cite{apc220_datasheet} es un transceptor de radiofrecuencia desarrollado por Appcon Technologies que permite establecer comunicaciones inalámbricas punto a punto en sistemas embebidos de bajo consumo.
    Este módulo opera en la banda ISM de 433 MHz lo que hace que no necesite licencia para funcionar.

    El APC220 emplea modulación GFSK (Gaussian Frequency Shift Keying), una variante de FSK (Frequency Shift Keying) en la que las transiciones de frecuencia entre los niveles binarios (0 y 1) se suavizan usando un filtro gaussiano aplicado previamente a la señal digital.
    Esta técnica reduce la anchura de banda de la señal transmitida y minimiza las emisiones fuera de banda, lo que se traduce en menor interferencia con otros sistemas y una mayor eficiencia espectral.
    Además, GFSK mejora la robustez frente al ruido, mejorando su eficiencia en entornos con interferencias.

    La tasa de transmisión es configurable entre 1200 bps y 19200 bps, utilizando UART como interfaz con el microcontrolador.

    El módulo integra un amplificador de potencia y un receptor de alta sensibilidad, permitiendo un alcance de hasta 1000 metros en condiciones óptimas con línea de visión clara.
    Además, incluye un circuito interno de corrección de errores y control automático de ganancia (AGC), que mejoran la fiabilidad de la comunicación.

    Su configuración puede ajustarse mediante comandos AT o mediante una herramienta software proporcionada por el fabricante, conectando el módulo a través de un convertidor USB-UART
\end{itemize}
\begin{table}[h]
    \centering
    \footnotesize
    \begin{tabular}{|l|c|c|c|}
        \hline
        \textbf{Característica} & \textbf{LoRa}                   & \textbf{XBee (802.15.4)} & \textbf{APC220}    \\
        \hline
        Frecuencia de operación & 433/868/915 MHz                 & 2.4 GHz                  & 433 MHz            \\
        \hline
        Modulación              & CSS (Chirp Spread Spectrum)     & DSSS + O-QPSK            & GFSK               \\
        \hline
        Velocidad de datos      & 0.3–27 kbps                     & 250 kbps                 & 1.2–19.2 kbps      \\
        \hline
        Alcance típico          & Hasta 15 km (entornos abiertos) & 30–300 m                 & Hasta 1 km         \\
        \hline
        Corrección de errores   & FEC configurable (CR 4/5–4/8)   & No especificado          & FEC + AGC internos \\
        \hline
        Interfaz con MCU        & UART                            & UART                     & UART               \\
        \hline
        Topología de red        & Punto a punto                   & Punto a punto / estrella & Punto a punto      \\
        \hline
    \end{tabular}
    \caption{Comparativa de tecnologías de comunicación por radiofrecuencia}
    \label{tab:comparativa_rf}
\end{table}


\section{Sensores empleados para telemetría: presión barométrica, IMU y GNSS}
A continuación se describen de los distintos tipos sensores empleados para cumplir con los requisitos del CanSat.
Estos sensores son los encargados de tomar mediciones precisas sobre distintas variables del entorno.
Para cumplir con estos requisitos se requiere la integración de tres tipos distintos de sensores:
\begin{itemize}
    \item \textbf{Presión barométrica:} Los sensores de presión barométrica se utilizan para medir la presión atmosférica y de esta forma estimar la altura sobre el nivel del mar mediante modelos estándar como el ISA (International Standard Atmosphere)~\cite{skybrary_isa}.

    La presión atmosférica es la fuerza que ejerce el peso de una columna de aire sobre un área determinada,
    por ello, al medir la presión sobre un punto de la tierra con mayor altitud la presión será menor por ser menor la cantidad de aire sobre ese punto.

    Distintos factores meteorológicos pueden hacer variar la presión atmosférica, principalmente con los cambios de temperatura,
    al variar la temperatura, varía la densidad del aire y, por lo tanto, varía su peso, afectando a la presión.

    Otros factores como la humedad relativa o el viento también influyen aunque de menor manera y pueden ser obviados.

    En el mercado se pueden encontrar distintos modelos de estos sensores, que varían entre ellos en función de su precisión, interfaz de conexión con el microcontrolador, consumo energético y precio.
    \begin{table}[h]
        \centering
        \footnotesize
        \begin{tabular}{|l|c|c|c|c|c|}
            \hline
            \textbf{Sensor} & \textbf{Rango de presión} & \textbf{Precisión} & \textbf{Interfaz}          & \textbf{Consumo típico} & \textbf{Precio (€)} \\
            \hline
            BMP388          & 300–1250 hPa              & ±8 Pa (±0.66 m)    & I\textsuperscript{2}C, SPI & 3.4 µA                  & \textasciitilde3.50 \\
            \hline
            BME280          & 300–1100 hPa              & ±12 Pa (±1 m)      & I\textsuperscript{2}C, SPI & 2.7 µA                  & \textasciitilde4.00 \\
            \hline
            MPL3115A2       & 50–1100 hPa               & ±0.04 hPa (±0.3 m) & I\textsuperscript{2}C      & 40 µA                   & \textasciitilde6.00 \\
            \hline
        \end{tabular}
        \caption{Comparativa de sensores de presión barométrica}
        \label{tab:barometric_sensors}
    \end{table}


    \item \textbf{IMU:} Una unidad de medición inercial o IMU, es un dispositivo que mide la velocidad, orientación y fuerzas gravitacionales de un aparato,
    para hacerlo generalmente utiliza una combinación de sensores, acelerómetros y giroscopios.

    Para su funcionamiento utiliza tres acelerómetros colocados de tal forma que sus ejes de medición queden de manera ortogonal entre sí,
    estos acelerómetros se utilizan para medir las fuerzas de aceleración (fuerzas G) que actúan en cada eje.

    También cuenta con tres giróscopios colocados en un patrón ortogonal similar, estos giróscopios miden la velocidad angular alrededor de cada eje, permitiendo conocer los cambios en la orientación del objeto respecto a un sistema de referencia.

    Algunos modelos también pueden contar con un magnetómetro, que mide el campo magnético terrestre y permite estimar el rumbo respecto al norte magnético.

    La combinación de estos sensores nos permite conocer la orientación en el espacio de un objeto en términos de los tres ángulos de Euler: pitch, yaw, y roll muy utilizados en aplicaciones aeronáuticas y espaciales.
    \begin{figure}[h]
        \centering
        \includegraphics[width=0.8\textwidth]{Imagenes/Bitmap/pitch_yaw_roll}
        \caption{Sistema de referencia basado en los ángulos de pitch, yaw y roll}
        \label{fig:pitch_yaw_roll}
    \end{figure}

    En algunos modelos de sensores disponibles en el mercado, esta combinación de sensores viene acompañada de un procesador interno que realiza fusión sensorial,
    permitiendo obtener directamente la orientación del dispositivo sin necesidad de cálculos externos.

    \begin{table}[h]
        \centering
        \resizebox{1.15\textwidth}{!}{
            \begin{tabular}{|l|c|c|c|c|c|}
                \hline
                \textbf{Sensor} & \textbf{Componentes}                     & \textbf{Salida de orientación}        & \textbf{Interfaz} & \textbf{Consumo típico} & \textbf{Precio (€)} \\
                \hline
                MPU6050         & Acelerómetro + Giroscopio                & No (requiere procesamiento externo)   & I\textsuperscript{2}C & 3.9 mA & \textasciitilde1.50 \\
                \hline
                BNO055          & Acelerómetro + Giroscopio + Magnetómetro & Sí (procesamiento interno con fusión) & I\textsuperscript{2}C / UART & 12 mA & \textasciitilde9.00 \\
                \hline
                BNO085          & Acelerómetro + Giroscopio + Magnetómetro & Sí (mejor precisión que BNO055)       & I\textsuperscript{2}C / UART / SPI & 3.5 mA & \textasciitilde14.00 \\
                \hline
                LSM9DS1         & Acelerómetro + Giroscopio + Magnetómetro & No (requiere fusión externa)          & I\textsuperscript{2}C / SPI & 1.0 mA & \textasciitilde5.00 \\
                \hline
            \end{tabular}
        }
        \caption{Comparativa de sensores IMU}
        \label{tab:imu_comparison}
    \end{table}

    \item \textbf{GNSS:} Global Navigation Satellite System~\cite{inertiallabs_gnss} es un sistema de navegación por satélite que engloba las siguientes constelaciones de satélites:
    GPS (Estados Unidos)~\cite{gps_system}, Galileo (Europa)~\cite{galileo_system}, GLONASS (Rusia)~\cite{glonass_system}y BeiDou (China)~\cite{beidou_system}.
    \begin{table}[h]
        \centering
        \footnotesize
        \begin{tabular}{|l|c|c|c|c|}
            \hline
            \textbf{Sistema} & \textbf{País / Región} & \textbf{Satélites operativos} & \textbf{Precisión típica (civil)} & \textbf{Frecuencias principales} \\
            \hline
            GPS              & Estados Unidos         & 31                            & 5–10 metros                       & L1, L2, L5                       \\
            \hline
            Galileo          & Unión Europea          & 28                            & <1 metro (con E5 AltBOC)          & E1, E5a, E5b, E6                 \\
            \hline
            GLONASS          & Rusia                  & 24                            & 5–10 metros                       & L1, L2                           \\
            \hline
            BeiDou           & China                  & 45                            & 2.5–5 metros                      & B1, B2, B3                       \\
            \hline
        \end{tabular}
        \caption{Comparativa de las principales constelaciones GNSS}
        \label{tab:gnss_constellations}
    \end{table}


    Estos sistemas se utilizan para determinar la posición geográfica y la velocidad de un objeto en cualquier parte del planeta utilizando estas constelaciones de satélites.
    Para determinar la posición y velocidad se necesita un receptor GNSS que recibe las señales emitidas por los satélites que contienen información sobre su posición y la fecha exacta en la que fueron enviadas.
    Cuando el receptor recibe señales de al menos tres satélites, puede calcular su posición usando una técnica geometrica llamada trilateración~\cite{trilateracion}.

    \begin{figure}[h]
        \centering
        \includegraphics[width=0.7\textwidth]{Imagenes/Bitmap/trilateracion}
        \caption{Ejemplo de cálculo de la posición usando la técnica trilateración}
        \label{fig:trilateracion}
    \end{figure}

    Esta técnica permite conocer la posición de un objeto conociendo su distancia a tres puntos de referencia,
    en este caso, estos tres puntos son las posiciones de los satélites.

    Actualmente, en el mercado existe una gran variedad de receptores GNSS diseñados para aplicaciones embebidas, que varían en su precisión, consumo, constelaciones soportadas y coste.
    A continuación se muestra una comparativa entre algunos de los receptores GNSS más comunes.
    \begin{table}[h]
        \centering
        \footnotesize
        \resizebox{1.15\textwidth}{!}{
            \begin{tabular}{|l|c|c|c|c|c|}
                \hline
                \textbf{Módulo} & \textbf{Constelaciones soportadas}      & \textbf{Precisión típica} & \textbf{Frecuencia de actualización} & \textbf{Consumo} & \textbf{Precio aprox.} \\
                \hline
                BN-880          & GPS, GLONASS, Galileo, BeiDou           & ~2m CEP                   & 1--10Hz                              & ~50mA a 5V       & ~15--25€               \\
                \hline
                NEO-M8N         & GPS, GLONASS, Galileo, BeiDou           & ~2m CEP                   & hasta 18Hz                           & <150mA a 5V      & ~35--40€               \\
                \hline
                NEO-F9P         & GPS, GLONASS, Galileo, BeiDou (con RTK) & cm-level con RTK          & hasta 20Hz                           & ~100mA a 3.3V & ~110--130€ \\
                \hline
            \end{tabular}
        }
        \caption{Comparativa de receptores GNSS comunes en sistemas embebidos}
        \label{tab:gnss_modules}
    \end{table}


\end{itemize}


\section{Captura y transmisión de vídeo en tiempo real}


\section{Visualización de datos en tiempo real: arquitecturas orientadas a eventos}
