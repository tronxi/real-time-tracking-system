\chapter{Diseño e implementación del sistema}


\section{Arquitectura general del sistema}


\section{Selección de componentes y justificación}


\section{Montaje electrónico del CanSat}


\section{Código embebido en la Raspberry Pi}


\section{Backend con Spring Boot}


\section{Frontend con Flutter para visualización en tiempo real}


\section{Pruebas de integración y validación}

