\chapter*{Resumen}


\section*{\tituloPortadaVal}

Este trabajo tiene como objetivo principal el desarrollo de una plataforma de visualización de datos en tiempo real para proyectos tipo CanSat que permita ser reutilizada en diferentes proyectos con diferente hardware.
Para validar la solución propuesta, se ha construido un prototipo funcional basado en un CanSat, que integra diferentes tipos de sensores, módulo GNSS, transmisión por radiofrecuencia, camara y además se integra con la plataforma.

La plataforma desarrollada se estructura en distintos módulos independientes (adquisición de datos, transmisión, backend, frontend) que pueden adaptarse fácilmente a nuevas configuraciones sin necesidad de modificar el sistema en su conjunto.
Este enfoque facilita la modificación del sistema y su integración en otros contextos con requisitos similares.

Durante el desarrollo se han cubierto todas las etapas clave, desde la adquisición y tratamiento de los datos hasta su visualización en tiempo real.
Se han realizado pruebas de funcionamiento que demuestran tanto la robustez de la arquitectura como la utilidad práctica del sistema.


\section*{Palabras clave}

\noindent CanSat, Visualización de datos, Arquitectura modular, Plataforma IoT, Telemetría, LoRa, WebSocket


   


